\documentclass[xcolor=dvipsnames]{beamer}
\usepackage[utf8]{inputenc}
\usepackage{graphicx}
\usepackage{amsmath}
\usepackage{amsfonts}
\usepackage{amssymb}
\usepackage{changepage}
\usepackage{subcaption}
\usepackage[square,numbers]{natbib}
\bibliographystyle{abbrvnat}
%\usepackage[backend=bibtex]{biblatex}
%\usepackage[backend=bibtex,style=numeric,defernumbers=true]{biblatex}
%\setbeamertemplate{bibliography item}{\insertbiblabel}
%\bibliography{references.bib}
%\usepackage{enumitem}
\usepackage[export]{adjustbox}
\usepackage{wrapfig}
\usepackage[belowskip=2pt,aboveskip=0pt]{caption}
\usepackage{media9}
\usepackage{hyperref}
\usepackage{lmodern}
\usepackage{multimedia}
%\usepackage{utopia} %font utopia imported
\usepackage{listings}
\lstloadlanguages{C,C++,csh,Java}

\definecolor{red}{rgb}{0.6,0,0} 
\definecolor{blue}{rgb}{0,0,0.6}
\definecolor{green}{rgb}{0,0.8,0}
\definecolor{cyan}{rgb}{0.0,0.6,0.6}
\lstset{
language=csh,
basicstyle=\script\ttfamily,
numbers=left,
numberstyle=\tiny,
numbersep=5pt,
tabsize=1,
extendedchars=true,
breaklines=true,
stringstyle=\color{blue}\ttfamily,
showspaces=false,
showtabs=false,
xleftmargin=17pt,
framexleftmargin=17pt,
framexrightmargin=1pt,
framexbottommargin=1pt,
commentstyle=\color{green},
morecomment=[l]{//}, %use comment-line-style!
morecomment=[s]{/*}{*/}, %for multiline comments
showstringspaces=false,
morekeywords={ SubMatrix abstract, event, new, struct,
as, explicit, null, switch,
base, extern, object, this,
bool, false, operator, throw,
break, finally, out, true,
byte, fixed, override, try,
case, float, params, typeof,
catch, for, private, uint,
char, foreach, protected, ulong,
checked, goto, public, unchecked,
class, if, readonly, unsafe,
const, implicit, ref, ushort,
continue, in, return, using,
decimal, int, sbyte, virtual,
default, interface, sealed, volatile,
delegate, internal, short, void,
do, is, sizeof, while,
double, lock, stackalloc,
else, long, static, SetSubMatrix,
enum, namespace, string},
keywordstyle=\color{cyan},
identifierstyle=\color{blue},
backgroundcolor=\color{white}
}
\usetheme{Madrid}
\usecolortheme{default}

\newcommand{\CS}{\textrm{C}\nolinebreak\hspace{-.05em}\raisebox{.6ex}{\tiny\bf \#}} 
\newcommand*{\myfont}{\fontfamily{phv}\selectfont} 

%gets rid of bottom navigation bars
\setbeamertemplate{footline}[page number]{}

%gets rid of navigation symbols
\setbeamertemplate{navigation symbols}{}

%------------------------------------------------------------
%This block of code defines the information to appear in the
%Title page
\title[NUMPDE] %optional
{\textbf{PDE BASED IMAGE RECONSTRUCTION}}

%\subtitle{A short story}

\author[Rosado, James] % (optional)
{\large{\textbf{James Rosado}}}

\date% (optional)
{\large{\textbf{ Numerical Partial Differential Equations\\ Spring 2020}}}

%End of title page configuration block
%------------------------------------------------------------

%------------------------------------------------------------
%The next block of commands puts the table of contents at the 
%beginning of each section and highlights the current section:
\titlegraphic{\vspace{0mm}\includegraphics[width=4.5cm]{Temple_University_logo.png}}
%\AtBeginSection[]
%{
%  \begin{frame}
%    \frametitle{Table of Contents}
%    \tableofcontents[currentsection]
%  \end{frame}
%}
%------------------------------------------------------------
\definecolor{MyBackground}{RGB}{231,231,231}
\setbeamercolor{background canvas}{bg=MyBackground}

\begin{document}

%The next statement creates the title page.

\frame{\titlepage}


%---------------------------------------------------------
%This block of code is for the table of contents after
%the title page


\begin{frame}
\frametitle{Table of Contents}
{\Large
\tableofcontents}
\end{frame}

%---------------------------------------------------------


\section{Introduction}
\begin{frame}{Project Goals}
    \begin{itemize}
        \item Implement an active contour method to delineate an image from noise.
        \item Determine necessary parameters to improve the detection of image contours and investigate the robustness of the active contour method. 
    \end{itemize}
\end{frame}

\begin{frame}{Project Steps}
\begin{itemize}
    \item Read in a raw image and process the image through Gaussian Smoothing with variance $\sigma$ \cite{osher2006level}.
    \item Set an edge detector function $g(\cdot)$ where $\displaystyle \lim_{|\vec{z}|\rightarrow\infty}g(\vec{z})=0$ and choose parameters \cite{osher2006level}
    \item Apply a level set method to move an active contour (a front, snake) towards the image using the image curvature and edge detector.
\end{itemize}
\end{frame}
\section{Analytic Formulation}
\begin{frame}{Active Contour Model}
\begin{itemize}
\item Raw image $u_0(x)$, where $x\in\Omega\subset\mathbb{R}^2$ in 2D
    \item Active contour $C$ minimizes:
    \begin{equation}\label{eqn1}
F(C)=\underbrace{\alpha\int_0^1|C'(s)|^2\ ds + \beta\int_0^1|C''(s)|\ ds}_{\textrm{control smoothness}} - \underbrace{\lambda\int_0^1|\nabla u_0(C(s))|^2\ ds}_{\textrm{controls movement of $C$}}
\end{equation}
\item $C$ realizes the image
\item Move $C$ using edge detector $g(\cdot)$ and curvature $\kappa$
\item Edge Detector \cite{osher2006level}, two parameters $\gamma$ and $p$
\begin{equation}\label{eqn2}
    g(\nabla u_0(\Vec{x}))=\frac{1}{1+\gamma|J_\sigma\star \nabla u_0|^p}
\end{equation}
\item $J_\sigma$ is the Gaussian Smoothing, but why?
\end{itemize}
\end{frame}
\begin{frame}{Active Contour Model}
    \begin{itemize}
        \item Equation (\ref{eqn1}) is equivalent to \cite{osher2006level,osti_478429} \begin{align}
\phi_t &= |\nabla\phi|\nabla\cdot\underbrace{\left[g(\nabla u_0) \left(\frac{\nabla\phi}{|\nabla\phi|}\right)\right]}_{\textrm{generalized curvature}}\notag \\[10pt]
&=|\nabla\phi|\left(g(\nabla u_0)\nabla\cdot\frac{\nabla\phi}{|\nabla\phi|}+\nabla g(\nabla u_0)\cdot\frac{\nabla\phi}{|\nabla\phi|} \right)\label{eqn3}\\[10pt]
&=|\nabla\phi|\left(g(\nabla u_0)\kappa+\nabla g(\nabla u_0)\cdot\frac{\nabla\phi}{|\nabla\phi|} \right)\notag \\[10pt]
\phi_t&=|\nabla\phi|g(\nabla u_0)\kappa+\nabla g(\nabla u_0)\cdot\nabla\phi\label{eqn4}
\end{align}
\item $\phi$ is the level set function
\item $\Gamma(t)$ is the contour corresponding to $\phi(x,t)=0$, which will outline our image.
\item Use (\ref{eqn3}) to write numerical approximation
    \end{itemize}
\end{frame}
\section{Gaussian Smoothing}
\begin{frame}{Smoothing An Image}
\begin{itemize}
    \item $\nabla u_0$ has order $\mathcal{O}(1/h)$ grows unbounded as $h\rightarrow 0$
\end{itemize}
\begin{figure}[h!]
  \centering
  \begin{overprint}
  \includegraphics<1>[width=\linewidth]{d1.png}
  \includegraphics<2>[width=\linewidth]{d2.png}
  \includegraphics<3>[width=\linewidth]{d3.png}
  \includegraphics<4>[width=\linewidth]{d4.png}
    \end{overprint}
\end{figure}
\end{frame}
\begin{frame}{Gaussian Smoothing}
\begin{itemize}
 \item Numerically solve heat equation up to time $t=\sigma$, $u_t=-\nabla^2u$
 \item Crank-Nicolson, two dimensions analogous to LeVeque \cite{leveque2007finite} \item $h_x=h_y=h$:
\[
\frac{U_{i,j}^{n+1}-U_{i,j}^n}{k}=\frac{1}{2}\left(\nabla^2U_{i,j}^n+\nabla^2U_{i,j}^{n+1}\right)
\]
where
\begin{align*}
    \nabla^2U_{i,j}^n&=\frac{U_{i-1,j}^n-2U_{i,j}^n+U_{i+1,j}^n}{h^2}+\frac{U_{i,j-1}^n-2U_{i,j}^n+U_{i,j+1}^n}{h^2}\\
    \nabla^2U_{i,j}^{n+1}&=\frac{U_{i-1,j}^{n+1}-2U_{i,j}^{n+1}+U_{i+1,j}^{n+1}}{h^2}+\frac{U_{i,j-1}^{n+1}-2U_{i,j}^{n+1}+U_{i,j+1}^{n+1}}{h^2}
\end{align*}
\item Constant Neumann boundary conditions
\end{itemize}
\end{frame}
\begin{frame}{Gaussian Smoothing}
\begin{figure}[h!]
  \centering
  \begin{overprint}
  \includegraphics<1>[width=\linewidth]{n1.png}
  \includegraphics<2>[width=\linewidth]{n2.png}
  \includegraphics<3>[width=\linewidth]{n3.png}
  \includegraphics<4>[width=\linewidth]{n4.png}
    \end{overprint}
\end{figure}
\end{frame}
\begin{frame}{Gaussian Smoothing}
    \begin{itemize}
        \item How are variance $\sigma$ and smoothing width $d$ related?
        \item analytically solve heat equation, Dirac initial data:
\[
\begin{cases}
u_t = -\nabla^2u & \vec{x}\in\Omega, t>0\\
u(x,0)= \delta_0(x)
\end{cases}
\]
\item Fundamental solution \cite{evans2010partial}:
\[
\Phi(\vec{x},t)=\frac{1}{4\pi t}e^{-|\vec{x}|^2/(4t)}
\]
 \item Amount of smoothing is determined by $t=\sigma>0$; therefore,
\[
\Phi(\vec{x},\sigma)=\frac{1}{4\pi \sigma}e^{-|\vec{x}|^2/(4\sigma)}
\]
    \end{itemize}
\end{frame}
\begin{frame}{Gaussian Smoothing}
\begin{itemize}
    \item Maximum when $\vec{x}=0$; therefore, $\Phi_{max}=1/(4\pi \sigma)$. \vspace{2mm}
    \item Width of the smoothing can be given by when $\Phi$ is $5\%$ of the maximum\vspace{2mm}
    \item solve
\[
0.05\cdot\frac{1}{4\pi\sigma}=\frac{1}{4\pi \sigma}e^{-d^2/(4\sigma)}
\]
which has solution given by $d(\sigma) = 2\sqrt{\sigma\ln(20)}$\vspace{2mm}
\item Grid resolution $h<d$ to resolve smoothing region.\vspace{2mm}
\item Choose $p$ and $\gamma$ such that $|\nabla g(\nabla u_0)|$ is order magnitude of $1/d$.
\end{itemize}
\end{frame}
\section{Numerical Solving}
\begin{frame}{Numerical Scheme}
\begin{itemize}
    \item Write out the terms for the \textit{generalized curvature} (\ref{eqn3}):
\[
\underbrace{A\nabla\cdot\frac{\nabla\phi}{|\nabla\phi|}}_{\textrm{Part 1}}+\underbrace{\nabla A\cdot\frac{\nabla\phi}{|\nabla\phi|}}_{\textrm{Part 2}}
\]
where $A = g(\nabla u_0)$
\item Part 1 
\begin{align*}
    &=A\left(\partial_x\left(\frac{\phi_x}{(\phi_x^2+\phi_y^2)^{1/2}}\right)+\partial_y\left(\frac{\phi_y}{(\phi_x^2+\phi_y^2)^{1/2}}\right)\right)\\
    &=A\left(\frac{|\nabla\phi|\phi_{xx}-\phi_x\frac{1}{|\nabla\phi|}(\phi_x\phi_{xx}+\phi_y\phi_{yx})}{|\nabla\phi|^2}+\cdots\right)\\
    &=\frac{A}{|\nabla\phi|^3}\left(|\nabla\phi|^2\phi_{xx}-\phi_x^2\phi_{xx}-2\phi_x\phi_y\phi_{xy}-\phi_y^2\phi_{yy}+|\nabla\phi|^2\phi_{yy}\right)
\end{align*}
\end{itemize}
\end{frame}
\begin{frame}{Numerical Scheme}
\begin{itemize}
    \item Use central finite differences to approximate first and second derivatives
    \item Part 2 use center differences for the gradient of $A$
    \[
    (A_x,A_y)\cdot\left(\frac{\phi_x}{\sqrt{\phi_x^2+\phi_y^2}^{1/2}},\frac{\phi_y}{\sqrt{\phi_x^2+\phi_y^2}^{1/2}}\right)
    \]
    \item Hamilton Jacobi Scheme in 2D
    \item Level set update is FE.
\end{itemize}
\end{frame}
\section{Results}
\begin{frame}{Run \#1: Some Noise}
\begin{figure}[h!]
  \centering
  \begin{overprint}
  \includegraphics<1>[width=\linewidth]{ex1original.png}
  \includegraphics<2>[width=\linewidth]{ex1blurred.png}
  \includegraphics<3>[width=\linewidth]{ex1spd.png}
    \end{overprint}
\end{figure}
\end{frame}
\begin{frame}{Run \# 1: Some Noise}
\movie[label=test,width=\linewidth, poster, showcontrols]{\includegraphics[width=\linewidth]{matlabLogo.png}}{vid1.mov}
\begin{itemize}
    \item $\sigma = 5.5\times 10^{-5}$, $p=8.9$, $\gamma = 5.5\times 10^{-11.25}$
\end{itemize}
\end{frame}
\begin{frame}{Run \#2: More Noise, same parameters}
    \begin{figure}[h!]
  \centering
  \begin{overprint}
  \includegraphics<1>[width=\linewidth]{ex2original.png}
  \includegraphics<2>[width=\linewidth]{ex2blurred.png}
  \includegraphics<3>[width=\linewidth]{ex2spd.png}
    \end{overprint}
\end{figure}
\end{frame}
\begin{frame}{Run \#2: More Noise, same parameters}
    \movie[label=test,width=\linewidth, poster, showcontrols]{\includegraphics[width=\linewidth]{matlabLogo.png}}{vid2.mov}
\begin{itemize}
    \item $\sigma = 5.5\times 10^{-5}$, $p=8.9$, $\gamma = 5.5\times 10^{-11.25}$
\end{itemize}
\end{frame}
\begin{frame}{Run \#3: More Noise, modified parameters}
    \begin{figure}[h!]
  \centering
  \begin{overprint}
  \includegraphics<1>[width=\linewidth]{ex3original.png}
  \includegraphics<2>[width=\linewidth]{ex3blurred.png}
  \includegraphics<3>[width=\linewidth]{ex3spd.png}
    \end{overprint}
\end{figure}
\end{frame}
\begin{frame}{Run \#3: More Noise, modified parameters}
     \movie[label=test,width=\linewidth, poster, showcontrols]{\includegraphics[width=\linewidth]{matlabLogo.png}}{vid3.mov}
\begin{itemize}
    \item $\sigma = 7.9\times 10^{-5}$, $p=7.0$, $\gamma = 5.5\times 10^{-9.55}$
\end{itemize}
\end{frame}

\begin{frame}{Run \#4: Spine-Dendrite Image}
    \begin{figure}[h!]
  \centering
  \begin{overprint}
  \includegraphics<1>[width=\linewidth]{ex4original.png}
  \includegraphics<2>[width=\linewidth]{ex4blurred.png}
  \includegraphics<3>[width=\linewidth]{ex4spd.png}
    \end{overprint}
\end{figure}
\end{frame}

\begin{frame}{Run \#4: Spine-Dendrite Image}
    \movie[label=test,width=\linewidth, poster, showcontrols]{\includegraphics[width=\linewidth]{matlabLogo.png}}{vid4.mov}
\begin{itemize}
    \item $\sigma = 5.5\times 10^{-5}$, $p=8.9$, $\gamma = 5.5\times 10^{-11.25}$
\end{itemize}
\end{frame}
\begin{frame}{Summary}
\textbf{Challenges}
\begin{itemize}
    \item Choosing parameters for reasonable image segmentation $p$ and $\gamma$
    \item Choosing the appropriate amount of smoothing without smoothing out important image features
\end{itemize}
\vspace{2mm}
\textbf{To be completed...}
\begin{itemize}
    \item Implement the Mumford-Shah functional \cite{doi:10.1002/cpa.3160430805}
\[
{\displaystyle E[J,B]=C\int _{D}(I({\vec {x}})-J({\vec {x}}))^{2}\,\mathrm {d} {\vec {x}}+A\int _{D/B}{\vec {\nabla }}J({\vec {x}})\cdot {\vec {\nabla }}J({\vec {x}})\,\mathrm {d} {\vec {x}}+B\int _{B}\ ds}
\]
optimizing this functional leads to a criteria for segmenting an image into sub-image regions and Ambrosio-Tortorelli give an algorithm for achieving the minimum and also show the minimum is well-defined.
    \item Implement on 3d geometries 
    \item establish concrete relationship between $\gamma$, $p$, and $d$.
    \item Use convolution to blurr image instead of solving heat equation
\end{itemize}
\end{frame}
\begin{frame}[allowframebreaks]
\frametitle{References}
{\tiny
\setbeamertemplate{bibliography item}[\theenumiv]
\nocite{*} 
%\bibliographystyle{amsalpha} 
\bibliography{references.bib}
}
\end{frame}

\end{document}